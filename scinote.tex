\documentclass{scinote}
\usepackage{tkz-graph}
\usetikzlibrary{calc}
\usetikzlibrary{positioning}


%%%%%%%%%%%%%%%%%%% biblatex %%%%%%%%%%%%%%%%%

%%%%%%%%%%%%%%%%%%%%% glossaries %%%%%%%%%%%%%%%%%
\setlength{\glsdescwidth}{1\linewidth}

\renewcommand\glossaryname{List of Abbreviations and Symbols}

\newglossaryentry{Q2}{name={$Q_2\left(f\right)$},
	%sort=Q2,
	description={Two-side (bounded) error quantum query complexity}}

\newglossaryentry{real_number}{name={$\mathbb{R}$},description={Real number}}

\newglossaryentry{v}{name={$\vec{v}$},description={a vector}}

% physics
\newglossaryentry{hamiltonian}{name={$\hat{H}$},description={Hamiltonian}}
\newglossaryentry{lagrangian}{name={$L$},description={Lagrangian}}

\newacronym{gcd}{GCD}{Greatest Common Divisor}

\newacronym{qft}{QFT}{Quantum Field Theory}
\newacronym{qm}{QM}{Quantum Mechanics}

% machine learning
\newacronym{gd}{GD}{Gradient Descent}
\newacronym{svm}{SVM}{Support Vector Machine}
\newacronym{mle}{MLE}{Maximum Likelihood Estimation}
\newacronym{map}{MAP}{Maximum A Posteriori Estimation}
\newacronym{pdf}{PDF}{Probability Density Function}
\newacronym{em}{EM}{Expectation Maximum}
\newacronym{gmm}{GMM}{Gaussian Mixture Model}
\newacronym{kl}{KL}{Kullback-Leibler}
\newacronym{vae}{VAE}{Variational Autoencoder}

% gan
\newacronym{gan}{GAN}{Generative Adversarial Nets}
\newacronym{cgan}{CGAN}{Conditional Generative Adversarial Nets}
\newacronym{dcgan}{DCGAN}{deep convolutional GAN}
\newacronym{wgangp}{WGAN-GP}{Wasserstein GAN with Gradient Penalty}

% autoregressvie model
\newacronym{rnn}{RNN}{Recurrent Neural Network}
\newacronym{lstm}{LSTM}{Long Short-Term Memory}
\newacronym{gru}{GRU}{Gated Recurrent Unit}

% diffusion models
\newacronym{ddm}{DDM}{Denoising Diffusion Models}
\newacronym{ddim}{DDIM}{Denoising Diffusion Implicit Models}

\newacronym{bert}{BERT}{Bidirectional Encoder Representations from Transformers}

\makeglossaries
%%%%%%%%%%%%%%%%%%%%% glossaries %%%%%%%%%%%%%%%%%

\usepackage{xargs}
\usepackage[colorinlistoftodos,prependcaption,textsize=tiny]{todonotes}
\newcommandx{\unsure}[2][1=]{\todo[linecolor=red,backgroundcolor=red!25,bordercolor=red,#1]{#2}}
\newcommandx{\change}[2][1=]{\todo[linecolor=blue,backgroundcolor=blue!25,bordercolor=blue,#1]{#2}}
\newcommandx{\info}[2][1=]{\todo[linecolor=purple,backgroundcolor=purple!25,bordercolor=purple,#1]{#2}}
\newcommandx{\improvement}[2][1=]{\todo[linecolor=green,backgroundcolor=green!25,bordercolor=green,#1]{#2}}
\newcommandx{\thiswillnotshow}[2][1=]{\todo[disable,#1]{#2}}

% math
\let\iff\relax
\newcommand{\iff}{\text{ iff }}
\newcommand{\OPT}{\textup{OPT}}

\DeclareRobustCommand\transp{^{\mathrm{T}}}

% lowercase italic for variables: x ($x$)
% lowercase italic bold for vectors: x ($\mathbold{x}$)
% uppercase italic bold for matrices: X ($\mathbold{X}$)
% uppercase italic for random variables: X ($X$)

\renewcommand{\vec}[1]{\ensuremath{\mathbold{#1}}}
\newcommand{\mat}[1]{\ensuremath{\mathbold{#1}}}
\newcommand{\vx}{\ensuremath{\vec{x}}}
\newcommand{\vX}{\ensuremath{\mat{X}}}

\DeclareMathOperator*{\argmax}{arg\!\max}
\DeclareMathOperator*{\argmin}{arg\!\min}

\makeatletter
\newcommand{\@giventhatstar}[2]{\left(#1\;\middle|\;#2\right)}
\newcommand{\@giventhatnostar}[3][]{#1(#2\;#1|\;#3#1)}
\newcommand{\giventhat}{\@ifstar\@giventhatstar\@giventhatnostar}
\makeatother

\newcommand{\pautoref}[1]{(\autoref{#1})}


%%%%%%%%%%%%%%%%%%%%%%%%%%%%%%%%%%%%%%%%%%%%%%%%%%
%%%%%%%%%%%%%%%% begin of document %%%%%%%%%%%%%%%
%%%%%%%%%%%%%%%%%%%%%%%%%%%%%%%%%%%%%%%%%%%%%%%%%%

\makeatletter
\newcommand{\email}[1]{%
  \def\@email{#1}%
}
\makeatother


\begin{document}

\title{\bf \huge Study Notes}
\author{Yangyang Li \\ \href{mailto:yangyang.li@northwestern.edu}{yangyang.li@northwestern.edu}}

\date{Update on \today}
\maketitle
\setcounter{tocdepth}{2}
\setcounter{minitocdepth}{1}

\begin{multicols}{2}
	\dominitoc% Initialization
	\adjustmtc[2]% chp number shift for mini-toc
	\tableofcontents
	\label{toc-contents}
\end{multicols}

\listoffigures
% \listoftables
\begin{multicols}{2}
	\listoftheorems[ignoreall,show={theorem}]
\end{multicols}

\renewcommand{\listtheoremname}{List of Definitions}
\begin{multicols}{2}
	\listoftheorems[ignoreall,show={definition}]
\end{multicols}

\printglossaries
% \printglossary[type=\acronymtype]
% \printglossary
% \printglossary[title=List of terms, toctitle=List of terms]

%https://github.com/cauliyang/Latex-Template-for-Scientific-Style-Book.git bib2gls
% \printunsrtglossaries % print all types
% \printunsrtglossary[type={abbreviations},title=List of Abbreviations,style=listgroup]
% \printunsrtglossary[type={abbreviations},title=List of Abbreviations,style=listhypergroup] % doesn't work
% \printunsrtglossary[type={symbols},title=List of Symbols,style=listgroup]
% \printunsrtglossary % main entry

%%%%%%%%%%%%%%%Content%%%%%%%%%%%%%%%
% \mainmatter % separat the number of toc and mainmatter
\chapter*{Preface}
\addcontentsline{toc}{chapter}{Preface}

\minitoc

% % \lipsum % dummy text - remove from real document

\section{Features of this template}
% \epigraph{\emph{... nature isn't classical, dammit, and if you want to make a simulation of nature, you'd better make it quantum mechanical, and by golly it's a wonderful problem, because it doesn't look so easy.}}{Richard Feynman (1981) Simulating physics with computers}
\epigraph{\emph{TeX, stylized within the system as \LaTeX, is a typesetting system which was designed and written by Donald Knuth and first released in 1978. TeX is a popular means of typesetting complex mathematical formulae; it has been noted as one of the most sophisticated digital typographical systems.}}{- \href{https://en.wikipedia.org/wiki/TeX}{Wikipedia}}

\subsection{crossref}
different styles of clickable definitions and theorems
\begin{itemize}
	\item nameref:
	      \nameref{def:gaussian_distribution}

	\item autoref:
	      \autoref{def:gaussian_distribution},
	      \autoref{alg:miller_rabin}

	\item cref:
	      \cref{def:gaussian_distribution},

	\item hyperref:
	      \hyperref[def:gaussian_distribution]{Gaussian},
\end{itemize}

\subsection{ToC (Table of Content)}
\begin{itemize}
	\item mini toc of sections at the beginning of each chapter
	\item list of theorems, definitions, figures
	\item the chapter titles are bi-directional linked
\end{itemize}

\subsection{header and footer}
fancyhdr
\begin{itemize}
	\item right header: section name and link to the beginning of the section
	\item left header: chapter title and link to the beginning of the chapter
	\item footer: page number linked to ToC of the whole document
\end{itemize}

\subsection{bib}
\begin{itemize}
	\item titles of reference is linked to the publisher webpage e.g., \cite{kitaev2002classical}
	\item backref (go to the page where the reference is cited) e.g., \cite{childsUniversalComputationQuantum2009}
	\item customized video entry in reference like in \cite{babaiGraphIsomorphismQuasipolynomial2016}
\end{itemize}

\subsection{preface, index, quote (epigraph) and appendix}
\myindex{index} page at the end of this document...

% \subsection{symbol and glossary (abbreviation)}
% examples:
% \gls{real_number},
% % \gls{natural_number},
% % \gls{complex_number},
% \gls{svm},
% \gls{v}

% \subsubsection{usage}
% \begin{itemize}
% 	\item glossary package
% 	      \begin{verbatim}
% 		pdflatex scinote.tex
% 		makeglossaries scinote
% 		pdflatex scinote.tex
% 	\end{verbatim}

% 	\item glossary-extra package and bib2gls
% 	      \begin{verbatim}
% 		pdflatex scinote.tex
% 		bib2gls scinote
% 		pdflatex scinote.tex
% 	\end{verbatim}
% \end{itemize}

% \section{Related Tools}
% \subsection{VSCode}
% Extension: \href{https://marketplace.visualstudio.com/items?itemName=James-Yu.latex-workshop}{Latex Workshop by James Yu}

% \subsubsection{settings}

% \subsection{lualatex and latexmk}
% .latexmkrc configuration file
% \begin{verbatim*}
% 	$pdflatex = 'lualatex -synctex=1 -interaction=nonstopmode --shell-escape %O %S';
% 		@generated_exts = (@generated_exts, 'synctex.gz');
% 	$pdf_mode = 1;

% 	add_cus_dep('glo', 'gls', 0, 'makeglo2gls');
% 	sub makeglo2gls {
% 			system("makeindex -s '$_[0]'.ist -t '$_[0]'.glg -o '$_[0]'.gls '$_[0]'.glo");
% 		}
% \end{verbatim*}
% To explain ....
% \begin{verbatim}
% # Also delete the *.glstex files from package glossaries-extra. Problem is,
% # that that package generates files of the form "basename-digit.glstex" if
% # multiple glossaries are present. Latexmk looks for "basename.glstex" and so
% # does not find those. For that purpose, use wildcard.
% $clean_ext = "%R-*.glstex";

% push @generated_exts, 'glstex', 'glg';

% add_cus_dep('aux', 'glstex', 0, 'run_bib2gls');

% # PERL subroutine. $_[0] is the argument (filename in this case).
% # File from author from here: https://tex.stackexchange.com/a/401979/120853
% sub run_bib2gls {
%     if ( $silent ) {
%     #    my $ret = system "bib2gls --silent --group '$_[0]'"; # Original version
%         my $ret = system "bib2gls --silent --group $_[0]"; # Runs in PowerShell
%     } else {
%     #    my $ret = system "bib2gls --group '$_[0]'"; # Original version
%         my $ret = system "bib2gls --group $_[0]"; # Runs in PowerShell
%     };

%     my ($base, $path) = fileparse( $_[0] );
%     if ($path && -e "$base.glstex") {
%         rename "$base.glstex", "$path$base.glstex";
%     }

%     # Analyze log file.
%     local *LOG;
%     $LOG = "$_[0].glg";
%     if (!$ret && -e $LOG) {
%         open LOG, "<$LOG";
%     while (<LOG>) {
%             if (/^Reading (.*\.bib)\s$/) {
%         rdb_ensure_file( $rule, $1 );
%         }
%     }
%     close LOG;
%     }
%     return $ret;
% }
% \end{verbatim}

% \subsection{Zotero and Better-bibtex}
% [todo]
% https://retorque.re/zotero-better-bibtex/
% customized entry, e.g., \textbf{Online Video}

% \section{Copyright and License}

% \begin{itemize}
% 	\item GitHub Repo: \url{https://github.com/cauliyang/Latex-Template-for-Scientific-Style-Book}
% 	\item Overleaf template: \url{https://www.overleaf.com/latex/templates/latex-template-for-scientific-style-book/ntprxjksmqxx}
% \end{itemize}


\part{Machine Learning}
% https://github.com/cauliyang/Machine-Learning-Session/blob/master/

\chapter{Probability}\label{chp:Probability}
\minitoc

\section{Maximum Likelihood Estimation}

\begin{equation}
	\vX = (\vx_1, \vx_2, \dots, \vx_N)\transp, \vx_i = (x_{i1}, x_{i2},\dots, x_{ip})\transp
\end{equation}
in which $N$ is the number of samples, $p$ is the number of features.
The data is sampled from a distribution $p\giventhat{\vx}{\theta}$, where $\theta$ is the parameter of the distribution.


For \(N\)  i.i.d. samples, the likelihood function is \(p \giventhat{\vX}{\theta} = \prod_{i=1}^{N} p \giventhat{\vx_i}{\theta}) \)

In order to get \(\theta\), we use \gls{mle}  to maximize the likelihood function.

\begin{equation}
	\theta_{\mathtt{MLE}} = \argmax_{\theta} \log p\giventhat{\vX}{\theta} = \argmax_{\theta} \sum_{i=1}^{N} \log p\giventhat{\vx_i}{\theta}
\end{equation}

\section{Maximum A Posteriori Estimation}
In Bayes' theorem, the \(\theta\) is not a constant value, but \(\theta \sim  p(\theta) \).
Hence,

\begin{equation}
	p\giventhat{\theta}{\vX} = \frac{p\giventhat{\vX}{\theta} p(\theta)}{p(\vX)}  =  \frac{p\giventhat{\vX}{\theta} p(\theta)}{\int\limits_{\theta} p\giventhat{\vX}{\theta} p(\theta) d\theta}
\end{equation}


In order to get \(\theta\), we use \gls{map}  to maximize the posterior function.

\begin{equation}
	\theta_{\mathtt{MAP}} = \argmax_{\theta} p\giventhat{\theta}{\vX} = \argmax_{\theta} \frac{p\giventhat{\vX}{\theta} p(\theta)}{p(\vX)}
\end{equation}


After \(\theta\) is estimated, then  calculating \(\frac{p\giventhat{\vX}{\theta} \cdot p(\theta)}{\int\limits_{\theta} p\giventhat{\vX}{\theta} p(\theta) d\theta}\) to get the posterior distribution.
We can use the posterior distribution to predict the probability of a new sample \(\vx\).

\begin{equation}
	p \giventhat{x_{\mathtt{new}}}{\vX}  = \int\limits_{\theta} p\giventhat{x_{\mathtt{new}}}{\theta} \cdot p\giventhat{\theta}{\vX} d\theta
\end{equation}

\section{Gaussian Distribution}

Gaussian distribution is also called normal distribution.

\begin{equation}
	\theta = (\mu, \sigma^2), \quad \mu = \frac{1}{N} \sum_{i=1}^{N} x_i, \quad \sigma^2 = \frac{1}{N} \sum_{i=1}^{N} (x_i - \mu)^2
\end{equation}

For \gls{mle},

\begin{equation}
	\theta = (\mu, \Sigma) = (\mu, \sigma^2), \quad \theta_{\mathtt{MLE}} = \argmax_{\theta} \log p\giventhat{\vX}{\theta} = \argmax_{\theta} \sum_{i=1}^{N} \log p\giventhat{\vx_i}{\theta}
\end{equation}


Generally, the \gls{pdf} of a Gaussian distribution is:

\begin{equation}
	p\giventhat{x}{\mu, \Sigma} =  \frac{1}{\sqrt{(2\pi)^p \det(\Sigma)}} \exp\left(-\frac{1}{2} (\vx - \mu)\transp \Sigma^{-1} (\vx - \mu)\right)
\end{equation}
in which \(\mu\) is the mean vector, \(\Sigma\) is the covariance matrix, \(\det\) is the determinant of matrix.
\(\det\)  is the product of all eigenvalues of a matrix.

Hence,

\begin{equation}
	\log p\giventhat{\vX}{\theta}  = \sum_{i=1}^{N} \log p\giventhat{x_i}{\theta} = \sum_{i=1}^{N} \log \frac{1}{\sqrt{(2\pi)^p \det(\Sigma)}} \exp\left(-\frac{1}{2} (\vx - \mu)\transp \Sigma^{-1} (\vx - \mu)\right)
\end{equation}

Let's only consider 1 dimension case for brevity, then

\begin{equation}
	\log p\giventhat{\vX}{\theta}  = \sum_{i=1}^{N} \log p\giventhat{x_i}{\theta} = \sum_{i=1}^{N} \log \frac{1}{\sqrt{2\pi \sigma^2}} \exp\left(-\frac{1}{2} \frac{(x - \mu)^2}{\sigma^2}\right)
\end{equation}

Let's get the optimal value for \(\mu\),

\begin{equation}
	\mu_{\mathtt{MLE}} = \argmax_{\mu} \log p\giventhat{\vX}{\theta} = \argmin_{\mu} \sum_{i=1}^{N} \frac{1}{2} \left(x_i - \mu\right)^2
\end{equation}

So,

\begin{equation}
	\frac{\partial \log p\giventhat{\vX}{\theta}}{\partial \mu} = \sum_{i=1}^{N} \left(\mu - x_i\right) = 0 \rightarrow \mu_{\mathtt{MLE}} = \frac{1}{N} \sum_{i=1}^{N} x_i
\end{equation}


Let's get the optimal value for \(\sigma^2\),

\begin{align*}
	\sigma_{\mathtt{MLE}} & = \argmax_{\sigma} \log p\giventhat{\vX}{\theta}                                                                                \\
	                      & =\argmax_{\sigma} \sum_{i=1}^{N} \log \frac{1}{\sqrt{2\pi \sigma^2}} \exp\left(-\frac{1}{2} \frac{(x - \mu)^2}{\sigma^2}\right) \\
	                      & = \argmax_{\sigma} \sum_{i=1}^{N} \left[ - \log\sqrt{2\pi \sigma^2} - \frac{(x - \mu)^2}{2\sigma^2} \right]                     \\
	                      & = \argmin_{\sigma} \sum_{i=1}^N \left[ \log \sigma + \frac{\left(x - \mu\right)^2}{2\sigma^2}\right]                            \\
\end{align*}

Hence,

\begin{equation}
	\frac{\partial}{\partial \sigma} \sum_{i=1}^N \left[\log \sigma + \frac{\left(x_i - \mu\right)^2}{2\sigma^2}\right] = 0 \rightarrow \sigma_{\mathtt{MLE}}^2 = \frac{1}{N} \sum_{i=1}^{N} \left(x_i - \mu\right)^2
\end{equation}

\(\mathbb{E}_{D}\left[\mu_{\mathtt{MLE}}\right]\)  is unbaised.

\begin{equation}
	\mathbb{E}_{D}\left[\mu_{\mathtt{MLE}}\right] =   \mathbb{E}_{D}\left[\frac{1}{N} \sum_{i=1}^{N} x_i\right] = \frac{1}{N} \sum_{i=1}^{N} \mathbb{E}_{D}\left[x_i\right] = \frac{1}{N} \sum_{i=1}^{N} \mu = \mu
\end{equation}

However, \(\mathbb{E}_{D}\left[\sigma_{\mathtt{MLE}}^2\right]\) is biased.

\begin{align}
	\mathbb{E}_{D}\left[\sigma_{\mathtt{MLE}}^2\right] & = \mathbb{E}_{D}\left[\frac{1}{N} \sum_{i=1}^{N} \left(x_i - \mu_{\mathtt{MLE}}\right)^2\right]                                                                                                                                                             \\
	                                                   & = \mathbb{E}_{D}\left[\frac{1}{N} \sum_{i=1}^{N}  \left(x_i - \mu_{\mathtt{MLE}}\right)^2\right]                                                                                                                                                            \\
	                                                   & = \mathbb{E}_{D}\left[\frac{1}{N} \sum_{i=1}^{N}  \left(x_i^2 - 2x_i \mu_{\mathtt{MLE}} + \mu_{\mathtt{MLE}}^2\right) \right] =  \mathbb{E}_{D}\left[\sum_{i=1}^{N} x_i^2 - 2 \frac{1}{N}\sum_{i=1}^{N}x_i \mu_{\mathtt{MLE}} + \mu_{\mathtt{MLE}}^2\right] \\
	                                                   & = \mathbb{E}_{D}\left[\frac{1}{N}\sum_{i=1}^{N} \left(x_i^2 - \mu^2\right) + \mu^2 - \mu_{\mathtt{MLE}}^2 \right]                                                                                                                                           \\
	                                                   & = \sigma^2 - \mathbb{E}_{D} \left[ \mu_{\mathtt{MLE}}^2 - \mu^2\right]                                                                                                                                                                                      \\
	                                                   & = \sigma^2 -  \left(\mathbb{E}_{D} \left[\mu_{\mathtt{MLE}}^2\right] - \mathbb{E}_{D} \left[\mu_{\mathtt{MLE}}^2\right] \right)                                                                                                                             \\
	                                                   & = \sigma^2 -  \mathtt{Var}\left[\mu_{\mathtt{MLE}}\right]  =  \sigma^2 - \mathtt{Var}\left[\frac{1}{N} \sum_{i=1}^N x_i \right]                                                                                                                             \\
	                                                   & = \sigma^2 - \frac{1}{N^2} \sum_{i=1}^N \mathtt{Var}\left[x_i\right] = \frac{N-1}{N} \sigma^2                                                                                                                                                               \\
\end{align}

\section{Hidden Markov Model}





\part{Algorithm and Data Structure}
\chapter{Algorithm}\label{chp:machine_learning}
\minitoc

\section{Graph}


\part{Programming}
\chapter{C++}\label{chp:c++}
\minitoc

\chapter{Rust}\label{chp:Rust}
\minitoc


\part{Research}
\chapter{Paper Reading}\label{chp:paper_reading}
\minitoc



\begin{appendices}
	\chapter{Formulas}

\section{Gaussian distribution}\label{sec:gaussian_distribution}

\begin{definition}[Gaussian distribution]\label{def:gaussian_distribution}
	\myindex{Gaussian distribution}
\end{definition}

\begin{theorem}[Central limit theorem]\label{thm:central_limit_theorem}
\end{theorem}


\end{appendices}

\backmatter

%%%%%%%%%%%%%%% Reference %%%%%%%%%%%%%%%

\printbibliography[heading=bibintoc]
\printindex

\end{document}

