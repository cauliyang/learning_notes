\chapter{Dive into Deep Learning}\label{chp:dive-into-deep-learning}

\section{Linear Neural Networks} % (fold)
\label{sec:Linear Neural Networks}

\subsection{Exercise 3.1.6}

\paragraph{Exercise 1:}

Let’s solve this analytically.

To minimize the sum of squared differences, we take the derivative of the function with respect to (b), set it equal to zero, and solve for (b).
The function to minimize is:

\[
	f(b) = \sum_{i=1}^{n} (x_i - b)^2
\]

Taking the derivative with respect to (b) gives:

$$
	f’(b) = \sum_{i=1}^{n} -2(x_i - b)
$$

Setting this equal to zero and solving for (b) gives:

$$
	0 = \sum_{i=1}^{n} -2(x_i - b)
$$
Solving for (b) gives:
$$
	b = \frac{1}{n} \sum_{i=1}^{n} x_i
$$

So the value of (b) that minimizes the sum of squared differences is the mean of the \(x_i\).

This problem and its solution relate to the normal distribution because the sum of squared differences is the basis for the maximum likelihood estimation of the mean of a normal distribution.
In other words, the mean of a normal distribution is the value that maximizes the likelihood of the observed data, which is equivalent to minimizing the sum of squared differences from the mean.


If we change the loss function to the sum of absolute differences, the problem becomes finding the median of the \(x_i\).
This is because the median is the value that minimizes the sum of absolute differences.
Unlike the mean, the median is not sensitive to extreme values, so it is a more robust measure of central tendency when there are outliers in the data.

\paragraph{Exercise 2:}

An affine function is a function composed of a linear transformation and a translation.
`In` the context of machine learning, an affine function often takes the form \(f(\mathbf{x}) = \mathbf{x}^\top \mathbf{w} + b)\), where \(\mathbf{x}\) is an input vector, \(\mathbf{w}\) is a weight vector, and (b) is a bias term.

We can show that this affine function is equivalent to a linear function on the augmented vector \((\mathbf{x}, 1)\) by considering a new weight vector \(\mathbf{w}' = (\mathbf{w}, b)\) and an augmented input vector \(\mathbf{x}' = (\mathbf{x}, 1)\). Then the affine function can be written as a linear function:

$$
	f(\mathbf{x}) = \mathbf{x}^\top \mathbf{w} + b = \mathbf{x}’^\top \mathbf{w}’
$$

Here’s the proof:

The original affine function is \(f(\mathbf{x}) = \mathbf{x}^\top \mathbf{w} + b\).

We define \(\mathbf{x}’ = (\mathbf{x}, 1)) and (\mathbf{w}’ = (\mathbf{w}, b)\).

The dot product \(\mathbf{x}’^\top \mathbf{w}’) is (\mathbf{x}^\top \mathbf{w} + 1 \cdot b\), which is exactly the original affine function.

Therefore, the affine function \(\mathbf{x}^\top \mathbf{w} + b\) is equivalent to the linear function \(\mathbf{x}'^\top \mathbf{w}'\) on the augmented vector \((\mathbf{x}, 1)\).
This equivalence is often used in machine learning to simplify the notation and the implementation of algorithms.

\paragraph{Exercise 3:}

A quadratic function of the form \(f(\mathbf{x}) = b + \sum_i w_i x_i + \sum_{j \leq i} w_{ij} x_{i} x_{j}\) can be implemented in a deep network using a fully connected layer followed by an element-wise multiplication operation and another fully connected layer.
Here’s how you can do it:

Input Layer: The input to the network is the vector \(\mathbf{x}\).

First Fully Connected Layer: This layer applies a linear transformation to the input vector.
The weights of this layer represent the \(w_i\) terms in the quadratic function. The output of this layer is \(b + \sum_i w_i x_i\).

Element-wise Multiplication Layer: This layer computes the element-wise product of the input vector with itself, resulting in the vector \(\mathbf{x} \odot \mathbf{x}\), where \(\odot\) denotes element-wise multiplication.
This operation corresponds to the \(x_i x_j\) terms in the quadratic function.

Second Fully Connected Layer: This layer applies a linear transformation to the output of the element-wise multiplication layer.
The weights of this layer represent the \(w_{ij}\) terms in the quadratic function. The output of this layer is \(\sum_{j \leq i} w_{ij} x_{i} x_{j}\).

Summation Layer: This layer adds the outputs of the first fully connected layer and the second fully connected layer to produce the final output of the network, which is the value of the quadratic function.

Note that this network does not include any activation functions, because the quadratic function is a polynomial function, not a nonlinear function.
If you want to model a more complex relationship between the input and the output, you can add activation functions to the network.

\paragraph{Exercise 4:}

If the design matrix \(\mathbf{X}^\top \mathbf{X}\) does not have full rank, it means that there are linearly dependent columns in the matrix \(\mathbf{X}\). This can lead to several problems:

The matrix \(\mathbf{X}^\top \mathbf{X}\) is not invertible, which means that the normal equations used to solve the linear regression problem do not have a unique solution.
This makes it impossible to find a unique set of regression coefficients that minimize the sum of squared residuals.

The presence of linearly dependent columns in \(\mathbf{X}\) can lead to overfitting, as the model can “learn” to use these redundant features to perfectly fit the training data, but it will not generalize well to new data.

One common way to fix this issue is to add a small amount of noise to the entries of \(\mathbf{X}\).
This can help to make the columns of \(\mathbf{X}\) linearly independent, which ensures that \(\mathbf{X}^\top \mathbf{X}\) has full rank and is invertible.
Another common approach is to use regularization techniques, such as ridge regression or lasso regression, which add a penalty term to the loss function that discourages the model from assigning too much importance to any one feature.

If you add a small amount of coordinate-wise independent Gaussian noise to all entries of \(\mathbf{X}\), the expected value of the design matrix \(\mathbf{X}^\top \mathbf{X}\) remains the same.
This is because the expected value of a Gaussian random variable is its mean, and adding a constant to a random variable shifts its mean by that constant. So if the noise has mean zero, the expected value of \(\mathbf{X}^\top \mathbf{X}\) does not change.

Stochastic gradient descent (SGD) can still be used when \(\mathbf{X}^\top \mathbf{X}\) does not have full rank, but it may not converge to a unique solution.
This is because SGD does not rely on the invertibility of \(\mathbf{X}^\top \mathbf{X}\), but instead iteratively updates the regression coefficients based on a randomly selected subset of the data.
However, the presence of linearly dependent columns in \(\mathbf{X}\) can cause the loss surface to have multiple minima, which means that SGD may converge to different solutions depending on the initial values of the regression coefficients.


\paragraph{Exercise 5:}

The negative log-likelihood of the data under the model is given by:

The likelihood function for the data is \(P(\mathbf y \mid \mathbf X) = \prod_{i=1}^{n} \frac{1}{2} \exp(-|y_i - (\mathbf x_i^\top \mathbf w + b)|)\), where \(y_i\) is the (i)-th target value, \(\mathbf x_i\) is the (i)-th input vector, \(\mathbf w\) is the weight vector, and (b) is the bias term.

Taking the negative logarithm of this gives the negative log-likelihood:

\(-\log P(\mathbf y \mid \mathbf X) = \sum_{i=1}^{n} \left( \log 2 + |y_i - (\mathbf x_i^\top \mathbf w + b)| \right)\).

Unfortunately, there is no closed-form solution for this problem.
The absolute value in the log-likelihood function makes it non-differentiable at zero, which means that we cannot set its derivative equal to zero and solve for \(\mathbf w\) and \(b\).

A minibatch stochastic gradient descent algorithm to solve this problem would involve the following steps:
Initialize \(\mathbf w\) and \(b\) with random values.
For each minibatch of data:
Compute the gradient of the negative log-likelihood with respect to \(\mathbf w\) and \(b\).
The gradient will be different depending on whether \(y_i - (\mathbf x_i^\top \mathbf w + b)\) is positive or negative.
Update \(\mathbf w\) and \(b\) by taking a step in the direction of the negative gradient.
Repeat until the algorithm converges.
One issue that could arise with this algorithm is that the updates could become very large if \(y_i - (\mathbf x_i^\top \mathbf w + b)\) is close to zero, because the gradient of the absolute value function is not defined at zero. This could cause the algorithm to diverge.

One possible solution to this problem is to use a variant of gradient descent that includes a regularization term, such as gradient descent with momentum or RMSProp.
These algorithms modify the update rule to prevent the updates from becoming too large.
Another solution is to add a small constant to the absolute value inside the logarithm to ensure that it is always differentiable.


\paragraph{Exercise 6:}

The composition of two linear layers in a neural network essentially results in another linear layer. This is due to the property of linearity: the composition of two linear functions is another linear function.

Mathematically, if we have two linear layers \(L_1\) and \(L_2\) such that \(L_1(x) = Ax + b\) and \(L_2(x) = Cx + d\), then the composition \(L_2(L_1(x)) = L_2(Ax + b) = C(Ax + b) + d = (CA)x + (Cb + d)\), which is another linear function.

This means that a neural network with two linear layers has the same expressive power as a neural network with a single linear layer. It cannot model complex, non-linear relationships between the input and the output.

To overcome this limitation, we typically introduce non-linearities between the linear layers in the form of activation functions, such as the ReLU (Rectified Linear Unit), sigmoid, or tanh functions. These non-linear activation functions allow the neural network to model complex, non-linear relationships and greatly increase its expressive power.

\paragraph{Exercise 7:}

Regression for realistic price estimation of houses or stock prices can be challenging due to the complex nature of these markets. Prices can be influenced by a wide range of factors, many of which may not be easily quantifiable or available for use in a regression model.

The additive Gaussian noise assumption may not be appropriate for several reasons.
First, prices cannot be negative, but a Gaussian distribution is defined over the entire real line, meaning it allows for negative values.
Second, the Gaussian distribution is symmetric, but price changes in real markets are often not symmetric.
For example, prices may be more likely to increase slowly but decrease rapidly, leading to a skewed distribution of price changes.
Finally, the Gaussian distribution assumes that large changes are extremely unlikely, but in real markets, large price fluctuations can and do occur.

Regression to the logarithm of the price can be a better approach because it can help to address some of the issues with the Gaussian noise assumption.
Taking the logarithm of the prices can help to stabilize the variance of the price changes and make the distribution of price changes more symmetric.
It also ensures that the predicted prices are always positive, since the exponential of any real number is positive.

When dealing with penny stocks, or stocks with very low prices, there are several additional factors to consider.
One issue is that the prices of these stocks may not be able to change smoothly due to the minimum tick size, or the smallest increment by which the price can change.
This can lead to a discontinuous distribution of price changes, which is not well modeled by a Gaussian distribution.
Another issue is that penny stocks are often less liquid than higher-priced stocks, meaning there may not be enough buyers or sellers to trade at all possible prices.
This can lead to large price jumps, which are also not well modeled by a Gaussian distribution.

The Black-Scholes model for option pricing is a celebrated model in financial economics that makes specific assumptions about the distribution of stock prices.
It assumes that the logarithm of the stock price follows a geometric Brownian motion, which implies that the stock price itself follows a log-normal distribution.
This model has been very influential, but it has also been criticized for its assumptions, which may not hold in real markets.
For example, it assumes that the volatility of the stock price is constant, which is often not the case in reality.

\paragraph{Exercise 8:}

The Gaussian additive noise model may not be appropriate for estimating the number of apples sold in a grocery store for several reasons:
The Gaussian model allows for negative values, but the number of apples sold cannot be negative.
The Gaussian model is a continuous distribution, but the number of apples sold is a discrete quantity.
The Gaussian model assumes that large deviations from the mean are extremely unlikely, but in reality, there could be large fluctuations in the number of apples sold due to various factors (e.g., seasonal demand, promotions).

The Poisson distribution is a discrete probability distribution that expresses the probability of a given number of events (in this case, the number of apples sold) occurring in a fixed interval of time or space.
The parameter \(\lambda\) is the rate at which events occur.
The expected value of a Poisson-distributed random variable is indeed \(\lambda\). This can be shown as follows:

The expected value (E[k]) of a random variable (k) distributed according to a Poisson distribution is given by:

$$E[k] = \sum_{k=0}^{\infty} k \cdot p(k \mid \lambda) = \sum_{k=0}^{\infty} k \cdot \frac{\lambda^k e^{-\lambda}}{k!}$$

By the properties of the Poisson distribution, this sum equals \(\lambda\).

The loss function associated with the Poisson distribution is typically the negative log-likelihood. Given observed counts \(y_1, y_2, \ldots, y_n\) and predicted rates \(\lambda_1, \lambda_2, \ldots, \lambda_n\), the negative log-likelihood is:

$$L(\lambda, y) = \sum_{i=1}^{n} (\lambda_i - y_i \log \lambda_i)$$

If we want to estimate \(\log \lambda\) instead of \(\lambda\), we can modify the loss function accordingly.
One possible choice is to use the squared difference between the logarithm of the predicted rate and the logarithm of the observed count:

$$L(\log \lambda, y) = \sum_{i=1}^{n} (\log \lambda_i - \log y_i)^2$$

This loss function penalizes relative errors in the prediction of the rate, rather than absolute errors. It can be more appropriate when the counts vary over a wide range, as it gives equal weight to proportional errors in the prediction of small and large counts.

\subsection{Exercise 3.3.5}

\paragraph{1. What will happen if the number of examples cannot be divided by the batch size. How would you change this behavior by specifying a different argument by using the framework’s API?}

PyTorch: In PyTorch's \emph{DataLoader}, there's an argument called \emph{drop\_last}.
If set to True, it will drop the last incomplete batch.
By default, it's set to \emph{False}.

\begin{minted}{python}
train_loader = torch.utils.data.DataLoader(dataset, batch_size=batch_size,
    shuffle=True, drop_last=True)
\end{minted}

\paragraph{2. Suppose that we want to generate a huge dataset, where both the size of the parameter vector w and the number of examples num\_examples are large.}

1. What happens if we cannot hold all data in memory?

2. How would you shuffle the data if it is held on disk? Your task is to design an efficient algorithm that does not require too many random reads or writes.
Hint: pseudorandom permutation generators 75 allow you to design a reshuffle without the need to store the permutation table explicitly (Naor and Reingold, 1999).

1. What happens if we cannot hold all data in memory?

If we cannot hold all the data in memory:

\textbf{Performance Impact}: Reading data directly from disk is much slower than reading from memory. If the training algorithm frequently accesses the disk, it can significantly slow down the training process.

\textbf{Out-of-Core Processing}: You'll need to use "out-of-core" or "external memory" algorithms, which are designed to process data that is too large to fit into a computer's main memory.
These algorithms break the data into smaller chunks that can fit into memory, process each chunk, and then combine the results.

\textbf{Streaming:} Data can be streamed from the disk in mini-batches.
Only one mini-batch is loaded into memory at a time, and after processing it, the next mini-batch is loaded.
This is common in deep learning when dealing with large datasets.

\textbf{Distributed Computing}: Another approach is to distribute the dataset across multiple machines in a cluster.
Each machine processes a subset of the data. Frameworks like TensorFlow and PyTorch have support for distributed training.

2. How would you shuffle the data if it is held on disk?

Shuffling large datasets on disk without too many random reads or writes can be challenging. Here's a method using pseudorandom permutation generators:

1. Pseudorandom Permutation: Use a pseudorandom permutation generator to generate a sequence of indices that represents the shuffled order of your data.
The beauty of pseudorandom permutation generators is that they can generate the nth element of the sequence without generating the first n-1 elements, allowing for efficient random access.

2. Sequential Reads and Writes:
- Read the data from the disk sequentially in large chunks (to benefit from sequential read speeds).
- For each item in the chunk, use the pseudorandom permutation generator to determine its new location in the shuffled dataset.
- Write the shuffled data back to the disk sequentially in large chunks.

3. In-Place Shuffling: If you have some memory available (but not enough to hold the entire dataset), you can load chunks of data into memory, shuffle them using the pseudorandom permutation, and then write them back to disk.
This reduces the number of disk writes.

4. External Sorting: Another approach is to use techniques from external sorting.
Divide the data into chunks that fit into memory, shuffle each chunk in memory, and then merge the chunks together in a shuffled manner.

5. Temporary Storage: If possible, use a fast intermediate storage (like an SSD) to temporarily store data chunks during the shuffling process.
This can speed up both reads and writes.

Remember, the key is to minimize random disk accesses, as they are much slower than sequential accesses.
Leveraging pseudorandom permutations allows for an efficient reshuffling without the need to store large permutation tables explicitly.


\paragraph{3. Implement a data generator that produces new data on the fly, every time the iterator is called.}

\begin{minted}{python}
import torch
from torch.utils.data import Dataset, DataLoader

class RandomDataset(Dataset):
    def __init__(self, num_samples, tensor_size):
        """
        Args:
            num_samples (int): Number of samples in the dataset.
            tensor_size (tuple): Size of the tensor to be generated.
        """
        self.num_samples = num_samples
        self.tensor_size = tensor_size

    def __len__(self):
        return self.num_samples

    def __getitem__(self, idx):
        # Generate a random tensor on the fly
        sample = torch.randn(self.tensor_size)
        return sample

# Parameters
num_samples = 1000
tensor_size = (3, 32, 32)  # Example: 3-channel, 32x32 image

# Create dataset and dataloader
random_dataset = RandomDataset(num_samples, tensor_size)
dataloader = DataLoader(random_dataset, batch_size=32, shuffle=True)

# Iterate over the dataloader
for batch in dataloader:
    print(batch.shape)  # Should print torch.Size([32, 3, 32, 32])
                        # for all batches except possibly the last one
\end{minted}

\paragraph{4. How would you design a random data generator that generates the same data each time it is called?}

\begin{minted}{python}
torch.manual_seed(idx)
\end{minted}


\subsection{Exercise 3.5.6}

\paragraph{1. How would you need to change the learning rate if you replace the aggregate loss over the minibatch with an average over the loss on the minibatch?}

When you replace the aggregate loss over the minibatch with an average over the loss on the minibatch, you're essentially scaling the loss by a factor of \( \frac{1}{\text{batch size}} \).
If you're using gradient descent or a variant thereof, the update rule is:

\[ \theta_{\text{new}} = \theta_{\text{old}} - \eta \nabla L \]

Where:
- \( \theta \) represents the parameters of the model.
- \( \eta \) is the learning rate.
- \( \nabla L \) is the gradient of the loss function \( L \) with respect to the parameters.

If you change the loss from an aggregate to an average, the gradient \( \nabla L \) will also be scaled by \( \frac{1}{\text{batch size}} \). To compensate for this scaling and to ensure that the magnitude of the parameter updates remains the same, you would need to multiply the learning rate \( \eta \) by the batch size.

In other words, if your original learning rate was \( \eta_{\text{original}} \) when using aggregate loss, and you switch to using average loss, your new learning rate should be:

\[ \eta_{\text{new}} = \eta_{\text{original}} \times \text{batch size} \]

However, in practice, many deep learning frameworks, including TensorFlow and PyTorch, compute the average loss over a minibatch by default.
So, if you're already using one of these frameworks, you likely don't need to adjust the learning rate when switching between aggregate and average loss, unless you were explicitly scaling the loss by the batch size yourself.

\paragraph{4. What is the effect on the solution if you change the learning rate and the number of epochs? Does it keep on improving?}

The learning rate and the number of epochs are two critical hyperparameters in training neural networks and other machine learning models. Their values can significantly impact the convergence and performance of the model. Let's discuss the effects of changing each:

1. Learning Rate (\( \eta \)):
- Too High: If the learning rate is set too high, the model might overshoot the minima in the loss landscape, leading to divergence. The loss might oscillate or even explode.
- Too Low: If the learning rate is too low, the model will converge very slowly. It might get stuck in local minima or plateaus in the loss landscape.
- Adaptive: Many modern optimizers (like Adam, RMSprop) adjust the learning rate dynamically based on recent gradients, which can help in finding a good balance.

2. Number of Epochs:
- Too Few: If the number of epochs is too low, the model might underfit the data, as it hasn't seen the data enough times to learn the underlying patterns.
- Too Many: If the number of epochs is too high, the model might overfit the data, especially if there isn't an early stopping mechanism or regularization in place. Overfitting means the model performs well on the training data but poorly on unseen data.

Does it keep on improving?

- Not necessarily. As training progresses:
- The loss typically decreases and the model's performance on the training set improves.
- However, after a certain point, the model might start to overfit, and its performance on a validation set might start to degrade.
- The optimal number of epochs is often where the performance on the validation set is maximized.

- The learning rate plays a crucial role in this:
- A good learning rate can help the model converge to a good solution faster.
- Learning rate schedules or decay can be beneficial. Starting with a larger learning rate (to progress quickly) and reducing it over time (to fine-tune) can be effective.

In summary, there's a balance to strike.
It's essential to monitor both training and validation metrics during training to determine the best values for the learning rate and the number of epochs.
Experimentation and hyperparameter tuning, possibly with techniques like cross-validation or Bayesian optimization, can help find the optimal values for a given problem and dataset.


\subsection{Exercise 4.1.5}

\paragraph{1. We can explore the connection between exponential families and softmax in some more depth}

1. Compute the second derivative of the cross-entropy loss \(l(y, \hat{y})\) for softmax.
2. Compute the variance of the distribution given by softmax(o) and show that it matches the second derivative computed above.

1. Second Derivative of Cross-Entropy Loss for Softmax, The cross-entropy loss for softmax is given by:

\[ l(y, \hat{y}) = - \sum_{i} y_i \log(\hat{y}_i) \]

Where \( \hat{y}_i \) is the softmax output:

\[ \hat{y}_i = \frac{\exp(o_i)}{\sum_j \exp(o_j)} \]

Let's compute the second derivative with respect to the logits \( o \).

First, the derivative of \( \hat{y}_i \) with respect to \( o_k \) is:

\[ \frac{\partial \hat{y}_i}{\partial o_k} = \hat{y}_i (\delta_{ik} - \hat{y}_k) \]

Where \( \delta_{ik} \) is the Kronecker delta (it's 1 when \( i = k \) and 0 otherwise).
Using the above, the second derivative is:

\[ \frac{\partial^2 l}{\partial o_i \partial o_k} = -\sum_j y_j \frac{\partial^2 \log(\hat{y}_j)}{\partial o_i \partial o_k} \]

\[ = -\sum_j y_j \frac{\partial}{\partial o_i} \left( \frac{\hat{y}_j (\delta_{jk} - \hat{y}_k)}{\hat{y}_j} \right) \]

\[ = -\sum_j y_j (\delta_{jk} - \hat{y}_k) \frac{\partial \hat{y}_j}{\partial o_i} \]

\[ = -\sum_j y_j (\delta_{jk} - \hat{y}_k) \hat{y}_j (\delta_{ji} - \hat{y}_i) \]

2. Variance of the Distribution Given by Softmax

The variance \( \text{Var}[X] \) of a discrete random variable \( X \) with possible outcomes \( x_1, x_2, \ldots \) and probabilities \( p_1, p_2, \ldots \) is:

\[ \text{Var}[X] = \sum_i p_i (x_i - \mathbb{E}[X])^2 \]

For the softmax distribution, \( X \) can take on values corresponding to the logits \( o \) with probabilities \( \hat{y} \). Thus, the variance becomes:

\[ \text{Var}[X] = \sum_i \hat{y}_i (o_i - \mathbb{E}[X])^2 \]

Where \( \mathbb{E}[X] = \sum_i \hat{y}_i o_i \).

Now, the second derivative of the cross-entropy loss with respect to the logits, as derived above, gives us a measure of the curvature of the loss.
When we compare this with the variance of the softmax distribution, we can see that they are related. The variance captures the spread of the logits, and the second derivative captures the sensitivity of the loss to changes in the logits. In the context of exponential families, this relationship between variance and the second derivative of the log-likelihood is a known property.

To show that they match, one would equate the expressions derived in the two sections above and simplify.
The detailed algebra can be quite involved, but the key insight is understanding the relationship between the curvature of the loss (second derivative) and the spread of the logits (variance).


\paragraph{5. Softmax gets its name from the following mapping: \(RealSoftMax(a, b) = log(exp(a) + exp(b))\)}

1. Prove that \( \text{RealSoftMax}(a, b) > \max(a, b) \):

Given:
\[ \text{RealSoftMax}(a, b) = \log(\exp(a) + \exp(b)) \]

Now, since \( \exp(a) \) and \( \exp(b) \) are both positive, we have:
\[ \exp(a) + \exp(b) > \exp(a) \]
\[ \exp(a) + \exp(b) > \exp(b) \]

Taking the logarithm of both sides (logarithm is a monotonically increasing function):
\[ \log(\exp(a) + \exp(b)) > \log(\exp(a)) = a \]
\[ \log(\exp(a) + \exp(b)) > \log(\exp(b)) = b \]

Thus, \( \text{RealSoftMax}(a, b) > \max(a, b) \).

2. How small can you make the difference between both functions?

Without loss of generality, let's set \( b = 0 \) and \( a \geq b \). Then:
\[ \text{RealSoftMax}(a, 0) = \log(\exp(a) + 1) \]

The difference between the two functions is:
\[ \text{RealSoftMax}(a, 0) - a = \log(\exp(a) + 1) - a \]

As \( a \) becomes large, the difference approaches 0, but it's always positive because of the "+1" inside the logarithm.

3. Prove that this holds for \( \lambda^{-1} \text{RealSoftMax}(\lambda a, \lambda b) \), provided that \( \lambda > 0 \):

Given:
\[ \lambda^{-1} \text{RealSoftMax}(\lambda a, \lambda b) = \lambda^{-1} \log(\exp(\lambda a) + \exp(\lambda b)) \]

Using the property of logarithms:
\[ = \log(\exp(\lambda a) + \exp(\lambda b))^{\lambda^{-1}} \]

Since \( \exp(\lambda a) \) and \( \exp(\lambda b) \) are both positive and \( \lambda > 0 \), the proof from part 1 applies here as well, and we can conclude that \( \lambda^{-1} \text{RealSoftMax}(\lambda a, \lambda b) > \max(a, b) \).

4. Show that for \( \lambda \to \infty \) we have \( \lambda^{-1} \text{RealSoftMax}(\lambda a, \lambda b) \to \max(a, b) \):

As \( \lambda \) becomes very large, the term \( \exp(\lambda a) \) will dominate \( \exp(\lambda b) \) if \( a > b \). Thus:
\[ \lambda^{-1} \log(\exp(\lambda a) + \exp(\lambda b)) \approx \lambda^{-1} \log(\exp(\lambda a)) = a \]

Similarly, if \( b > a \), the result will be \( b \). Therefore, as \( \lambda \to \infty \), \( \lambda^{-1} \text{RealSoftMax}(\lambda a, \lambda b) \to \max(a, b) \).

5. Construct an analogous softmin function:

The softmin function can be constructed using the negative of the inputs to the softmax function:
\[ \text{RealSoftMin}(a, b) = -\text{RealSoftMax}(-a, -b) \]
\[ = \log(\exp(-a) + \exp(-b)) \]

6. Extend this to more than two numbers:

For \( n \) numbers \( a_1, a_2, \ldots, a_n \):
\[ \text{RealSoftMax}(a_1, a_2, \ldots, a_n) = \log(\exp(a_1) + \exp(a_2) + \ldots + \exp(a_n)) \]

Similarly, the softmin for \( n \) numbers is:
\[ \text{RealSoftMin}(a_1, a_2, \ldots, a_n) = \log(\exp(-a_1) + \exp(-a_2) + \ldots + \exp(-a_n)) \]

\subsection{Exercise 4.5.5}

\paragraph{Deep learning uses many different number formats, including FP64 double precision (used extremely rarely), FP32 single precision, BFLOAT16 (good for compressed representations), FP16 (very unstable), TF32 (a new format from NVIDIA), and INT8. Compute the smallest and largest argument of the exponential function for which the result does not lead to numerical underflow or overflow}

\begin{table}[h!]
	\centering
	\begin{tabular}{lrrrrrrr}
		\toprule
		format         & format\_min & format\_max & exp\_min   & exp\_max & test\_min & exp(max-1) & exp(max+1) \\
		\midrule
		torch.float64  & -1.80e+308  & 1.80e+308   & -1.80e+308 & 7.10e+02 & 0.00e+00  & 6.61e+307  & inf        \\
		torch.float32  & -3.40e+38   & 3.40e+38    & -3.40e+38  & 8.87e+01 & 0.00e+00  & 1.25e+38   & inf        \\
		torch.bfloat16 & -3.39e+38   & 3.39e+38    & -3.39e+38  & 8.85e+01 & 0.00e+00  & 1.00e+38   & inf        \\
		torch.float16  & -6.55e+04   & 6.55e+04    & -6.55e+04  & 1.11e+01 & 0.00e+00  & 2.41e+04   & inf        \\
		torch.int8     & -1.28e+02   & 1.27e+02    & -1.28e+02  & 4.84e+00 & 0.00e+00  & 4.67e+01   & 8.90e+01   \\
		\bottomrule
	\end{tabular}
\end{table}

1. FP64 (Double Precision):
- Smallest positive normalized number \(2^{-1022}\)
- Largest finite representable number \( (2-2^{-52}) \times 2^{1023} \)
- For the exponential function, the range of \(x\) for which \(e^x\) does not overflow would be approximately \([-1022 \log(2), 1023 \log(2)]\).

2. FP32 (Single Precision):
- Smallest positive normalized number: \(2^{-126}\)
- Largest finite representable number: \( (2-2^{-23}) \times 2^{127} \)
- For the exponential function, the range of \(x\) for which \(e^x\) does not overflow would be approximately \([-126 \log(2), 127 \log(2)]\).

3. BFLOAT16:
- This format has the same exponent bits as FP32 but fewer precision bits.
- The range for the exponential function would be similar to FP32.

4. FP16:
- Smallest positive normalized number: \(2^{-14}\)
- Largest finite representable number: \( (2-2^{-10}) \times 2^{15} \)
- For the exponential function, the range of \(x\) for which \(e^x\) does not overflow would be approximately \([-14 \log(2), 15 \log(2)]\).

5. TF32:
- This is a new format introduced by NVIDIA, and it has 10 bits for the exponent and 19 bits for the fraction.
- The range would be somewhere between FP32 and FP16.

6. INT8:
- This is an integer format, so it doesn't directly apply to the exponential function in the same way as the floating-point formats. However, if used in quantized neural networks, the range would be \([-128, 127]\).

For the exact ranges, especially for the newer formats like TF32, one would need to refer to the official documentation or specifications provided by the manufacturers.

To avoid overflow or underflow when computing the exponential function, it's common to clip or bound the input values to the function within the safe range for the given number format.

\subsection{Exercise 7.1.6}

\paragraph{1. Assume that the size of the convolution kernel is \(\Delta = 0 \). Show that in this case the convolution kernel implements an MLP independently for each set of channels. This leads to the Network in Network architectures}

When the size of the convolution kernel is \( \Delta = 0 \), it means that the kernel effectively has a size of \( 1 \times 1 \).
A \( 1 \times 1 \) convolution kernel operates on each pixel individually, without considering its neighbors.

Let's break down the implications of this:

1. Independent Operation on Each Pixel: A \( 1 \times 1 \) convolution operates on each pixel independently. For an input with \( C_{in} \) channels, each pixel is represented by a \( C_{in} \)-dimensional vector. The \( 1 \times 1 \) convolution transforms this \( C_{in} \)-dimensional vector into a \( C_{out} \)-dimensional vector, where \( C_{out} \) is the number of output channels.

2. MLP for Each Set of Channels: The transformation from \( C_{in} \) to \( C_{out} \) for each pixel can be viewed as a fully connected layer (or MLP) applied independently to each spatial location. This is because the weights of the \( 1 \times 1 \) convolution kernel can be seen as the weights of a fully connected layer that connects each input channel to each output channel.

3. Network in Network (NiN): The idea of using \( 1 \times 1 \) convolutions to implement an MLP for each set of channels led to the Network in Network (NiN) architecture. In NiN, multiple \( 1 \times 1 \) convolutional layers are stacked, allowing for deep per-pixel transformations. This introduces additional non-linearity for each pixel without changing the spatial dimensions.

In summary, a \( 1 \times 1 \) convolution kernel effectively applies an MLP to each pixel independently, transforming the channels at each spatial location.
This concept is central to the Network in Network (NiN) architecture, which uses \( 1 \times 1 \) convolutions to introduce additional non-linearities in the network without altering the spatial dimensions.

\paragraph{5. What happens with convolutions when an object is at the boundary of an image?}

When an object is at the boundary of an image, and we apply convolutions, several issues and considerations arise:

1. Loss of Information: The primary issue is that for pixels on the boundary, there aren't enough neighboring pixels to apply the convolution kernel fully. This means that these boundary pixels might not be processed adequately, leading to potential loss of information about objects at the image boundary.

2. Padding: To address the boundary issue, one common approach is to add padding around the image. Padding involves adding extra pixels around the boundary of the image. There are different types of padding:
- Zero Padding: Adding pixels with a value of zero.
- Reflect Padding: The boundary pixels are reflected. For instance, if the last row of an image is \([a, b, c]\), it becomes \([a, b, c, c, b, a]\) after reflect padding.
- Replicate Padding: The boundary pixels are replicated. Using the previous example, it becomes \([a, b, c, c, c, c]\).

Padding ensures that every pixel, including those on the boundary, has a full neighborhood of pixels to apply the convolution kernel.

3. Valid vs. Same Convolutions:
- Valid Convolution: No padding is used. The resulting feature map after convolution is smaller than the input image because the kernel can only be applied to positions where it fits entirely within the image boundaries.
- Same Convolution: Padding is added such that the output feature map has the same spatial dimensions as the input image.

4. Edge Effects: Even with padding, objects at the boundary might be affected differently than objects in the center of the image. This is because the artificial values introduced by padding might not represent the actual image content. This can lead to edge artifacts in the output feature map.

5. Dilated Convolutions: In dilated convolutions, where the kernel elements are spaced out by introducing gaps, the boundary effects can be even more pronounced, especially if the dilation rate is high.

6. Pooling Layers: The boundary effects can also impact pooling layers (like max-pooling or average-pooling). If the pooling window doesn't fit entirely within the image or feature map boundaries, padding might be needed, or the pooling window might be truncated.

\paragraph{6. Prove that the convolution is symmetric, \(f * g = g * f\)}

To prove that convolution is symmetric, we'll start with the definition of the convolution operation for two functions \( f(t) \) and \( g(t) \):

\[ (f * g)(t) = \int_{-\infty}^{\infty} f(\tau) g(t - \tau) \, d\tau \]

Now, let's express the convolution of \( g \) and \( f \):

\[ (g * f)(t) = \int_{-\infty}^{\infty} g(\tau) f(t - \tau) \, d\tau \]

To show that these two expressions are equivalent, we'll make a substitution:

Let \( \lambda = t - \tau \)
This implies \( d\lambda = -d\tau \)

When \( \tau = -\infty \), \( \lambda = t + \infty = \infty \)
When \( \tau = \infty \), \( \lambda = t - \infty = -\infty \)

Substituting these values into the convolution of \( g \) and \( f \), we get:

\[ (g * f)(t) = \int_{\infty}^{-\infty} g(t - \lambda) f(\lambda) (-d\lambda) \]

Now, reversing the limits of integration:

\[ (g * f)(t) = \int_{-\infty}^{\infty} g(t - \lambda) f(\lambda) \, d\lambda \]

Comparing this with the expression for \( (f * g)(t) \), we see that the two expressions are equivalent:

\[ (f * g)(t) = (g * f)(t) \]

Thus, we've proven that the convolution operation is symmetric:

\[ f * g = g * f \]


\paragraph{Exercise 7.2.8}

\paragraph{4. How do you represent a cross-correlation operation as a matrix multiplication by changing the input and kernel tensors?}

Cross-correlation can be represented as a matrix multiplication by converting the input and kernel tensors into appropriate matrix and vector forms, respectively.
This process is often referred to as ``toeplitzization'' or ``im2col'' (image to column) transformation. Here's how you can achieve this:

1. Flatten the Kernel:
- First, you reshape the kernel into a column vector.
- If the kernel is of size \( k \times k \), the resulting vector will be of size \( k^2 \times 1 \).

2. Toeplitz Matrix for Input:
- For each position in the output, extract a \( k \times k \) patch from the input, and flatten it into a row vector.
- Stack these row vectors on top of each other to form a matrix. This matrix has a special structure called a Toeplitz matrix.
- If the input is of size \( n \times n \) and the kernel is \( k \times k \), then the resulting matrix will be of size \( \left(n-k+1\right)^2   \times k^2 \).

3. Matrix Multiplication:
- Multiply the Toeplitz matrix (from step 2) with the flattened kernel (from step 1). The result will be a column vector representing the flattened output of the cross-correlation operation.
- Reshape this column vector back into a 2D matrix to get the final output.

Mathematically, if \( M \) is the Toeplitz matrix derived from the input and \( K \) is the column vector derived from the kernel, then the output \( O \) is given by:

\[ O = M \times K \]

This representation is particularly useful for certain computational platforms and algorithms that can efficiently handle matrix multiplications.
In deep learning frameworks, this transformation is often used under the hood to speed up convolution operations, especially when using GPUs.

Note: This explanation assumes 2D inputs and kernels for simplicity, but the concept can be extended to higher dimensions.

\begin{minted}{python}
def im2col(inputs, kernel_size):
    # Extract patches from the input tensor
    k, _ = kernel_size
    unfold = torch.nn.Unfold(kernel_size=k, stride=1, padding=0)
    cols = unfold(inputs)
    return cols.transpose(1, 2)
\end{minted}

% section Linear Neural Networks (end)
