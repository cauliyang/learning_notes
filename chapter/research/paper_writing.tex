\chapter{Paper Writing}\label{chp:paper_writing}
\minitoc


\section{Resources}

\begin{itemize}
	\item \url{https://github.com/MLNLP-World/Paper-Writing-Tips}
	\item \url{https://github.com/dspinellis/latex-advice}
	\item \url{https://github.com/guanyingc/latex_paper_writing_tips}
	\item \url{https://github.com/yzy1996/English-Writing}
	\item \url{https://github.com/guanyingc/cv_rebuttal_template}
	\item \url{https://www.overleaf.com/gallery/tagged/tikz/page}
	\item \url{https://color.adobe.com/zh/explore}
	\item \url{https://github.com/acmi-lab/cmu-10717-the-art-of-the-paper/tree/main/resources}
	\item \url{https://github.com/daveshap/ChatGPT_Custom_Instructions/blob/main/academic_copy_reviser.md}
	\item \url{https://github.com/sylvainhalle/textidote?tab=readme-ov-file}
	\item \url{https://github.com/egeerardyn/awesome-LaTeX}
	\item \url{https://texfaq.org}
	\item \url{https://github.com/xinychen/awesome-latex-drawing}
	\item \url{https://pudding.cool/2022/02/plain}
	\item \url{https://www.youtube.com/watch?v=VK51E3gHENc&ab_channel=MicrosoftResearch}
	\item \url{https://www.cs.jhu.edu/~jason/advice/write-the-paper-first.html}
	\item \url{https://github.com/yzy1996/English-Writing}
\end{itemize}

\section{Writing Tools}

\begin{table}
	\caption{Writing Tools}\label{tab:writing_tools}
	\begin{tabular}{ll}
		\toprule
		\textbf{Tool}                     & \textbf{URL}                                       \\
		\midrule
		Latex Table Generator             & \url{tablesgenerator.com}                          \\
		Latex Equation Editor             & \url{https://www.codecogs.com/latex/eqneditor.php} \\
		Latex Equation and Figure Editor  & \url{https://www.mathcha.io/editor}                \\
		Equation Image to Latex           & \url{https://mathpix.com/}                         \\
		Figure Inspiration                & \url{https://echarts.apache.org/examples/}         \\
		Figure Creation                   & \url{https://altair-viz.github.io/}                \\
		Online Figure Editor              & \url{https://www.tldraw.com/}                      \\
		Identify Font                     & \url{https://www.fontsquirrel.com/matcherator}     \\
		Identify Color                    & \url{https://imagecolorpicker.com/}                \\
		Identify Math Symbol              & \url{https://detexify.kirelabs.org/classify.html}  \\
		Better Labeling Tool              & \url{https://labelstud.io/}                        \\
		Fix Wrong Citation                & \url{https://github.com/yuchenlin/rebiber}         \\
		Have a good name for model        & \url{http://acronymify.com/}                       \\
		Deadline Reminder                 & \url{https://aideadlin.es/?sub=ML,CV,NLP}          \\
		Great Paper/Model Info            & \url{https://paperswithcode.com/}                  \\
		Same Meaning Words                & \url{https://www.thesaurus.com/}                   \\
		Compare Same Meaning Words        & \url{https://wikidiff.com/neglect/omit}            \\
		Common Words for Academic Writing & \url{http://www.esoda.org/}                        \\
		Doi to Bib                        & \url{https://www.doi2bib.org/}                     \\
		Figure Gallery of Vega            & \url{https://vega.github.io/vega-lite/examples/}   \\
		Figure Gallery of Graphviz        & \url{https://graphviz.org/gallery/}                \\
		UML Diagram                       & \url{https://plantuml.com/}                        \\
		Color Picker                      & \url{https://colorbrewer2.org}                     \\
		Abstract/Title Generator          & \url{https://x.writefull.com/}                     \\
		Online BibTeX Formatter           & \url{https://flamingtempura.github.io/bibtex-tidy} \\
		\bottomrule
	\end{tabular}
\end{table}

\section{Writing Tips}

Your draft can describe your motivation, formal problem, model, algorithms, and experiments before you actually build anything.
(Ideally, it will also explain why you did it this way rather than some other way, and point out gaps that remain for future work.)
By showing others the draft at this stage, you'll get important feedback before you invest time in the~\enquote{wrong} work.

If you run into trouble while doing the work, then I may have difficulty diagnosing or even understanding the problem you are facing.
We may waste a whole meeting just getting aligned.
But if you can show me a precise writeup of what you're doing, then I can be much more helpful.
In general, meetings are very productive when we have a concrete document in front of us.


Of course, there are many kinds of concrete documents.
You could instead write up the notes from our (usually extensive) discussions as private documents for further discussion.
Still, writing up in the form of a final paper makes you (1) integrate everything in one place, (2) decide which ideas will be made central for this paper, and (3) focus on the coherence and impact of the end product.
(Additional discussion and brainstorming can still go into the document—those subsections can be cut or moved to an appendix for the conference paper, but retained for a longer version as a tech report, journal article, or dissertation.)\footnote{\url{https://www.cs.jhu.edu/~jason/advice/write-the-paper-first.html}}

\paragraph{What needs to be done?}
Assuming the idea is indeed a good one, then writing a draft makes you sharpen the message of the paper.
Then you can figure out what work needs to be done to support exactly that message.

\begin{itemize}
	\item Your introduction will make some claims. Often you will realize that some interesting additional experiment would really test those claims.
	\item Writing the literature review will help you design your experiments. It may influence what datasets you use, what comparisons you do, and what you are trying to prove you can do that other people can't.
	      However, don't write the lit review first. Write out your own ideas before comparing them with the literature. This increases the chance that you'll find new angles on the problem.
	\item Writing the experimental section is possible even before you've done the experiments. Explain the full experimental design. Make empty result tables with row and column headers and explanatory captions. Make empty graphs with axes and captions. The actual results will be missing, but it will be clear what work is necessary.
	      If you are having trouble switching your brain from experiments to writing, just write up your current experiments first. Although writing an intro will also help, so you can explain why you are doing the experiments.
\end{itemize}

Honest writing may lead you to realize that proving your point requires more work than you'd thought (which is why you'd better write early).


\paragraph{How to give a wonderful talk?}
\begin{itemize}
	\item What is the question?
	\item Why is it interesting?
	\item How do you propose to answer it?
\end{itemize}


\begin{itemize}
	\item e.g. means \enquote{for example}.
	\item i.e.  \ means \enquote{that is}.
	\item et al. means \enquote{and others of the same kind}.
	\item etc. means \enquote{and others}, do not specify people.
\end{itemize}

\paragraph{Avoid extreme meaning words}

\begin{table}[ht]
	\caption{Alternative Terms for Common Words}\label{tab:alternative_terms}
	\centering
	\begin{tabular}{@{}ll@{}}
		\toprule
		Original Term    & Alternative Terms                        \\
		\midrule
		obvious          & straightforward                          \\
		always           & usually, generally, often                \\
		never            & seldom, rare                             \\
		avoid, eliminate & reduce, mitigate, alleviate, relieve     \\
		a lot of         & much, many                               \\
		do (verb)        & perform, conduct, carry out              \\
		big              & large, significant                       \\
		like             & such as, for example                     \\
		think            & believe, consider, argue, claim, suggest \\
		talk             & discuss, describe, present, show         \\
		look at          & examine, investigate, explore, study     \\
		get              & obtain, acquire, receive, gain           \\
		keep             & maintain, retain, preserve               \\
		climb            & increase, rise, grow, improve, ascend    \\
		really           & very, extremely, highly                  \\
		\bottomrule
	\end{tabular}
\end{table}


\begin{enumerate}
	\item  Don’t spend too much time reading other people’s research, waiting for inspiration to strike you.
	      \begin{enumerate}
		      \item  Reading research should be a regular part of any economist’s routine.
		      \item But it is not a substitute for doing your own.
	      \end{enumerate}

	\item Don’t set the bar too high. You don’t need to win the Nobel.
	\item  But don’t set the bar too low!
	      \begin{enumerate}
		      \item Be wary of picking a topic that is of interest to a small number of people (e.g. you and your adviser).
		      \item In particular, think about topics that will interest those beyond this island
	      \end{enumerate}
	\item Your ideas and results won’t sell themselves.
	\item How you communicate your work is of crucial importance.
	\item There is no point in having an interesting piece of research that nobody understands or sees the point of.
	\item Many economists think of themselves as primarily experts in technical methods: Econometrics, economic theory, data expertise.
	\item This “white coat” mentality—that we are mainly scientists who then do a write-up of our results—is deeply wrong.
	\item Writing is an essential part of the research process, not a last-minute thing to be rushed.
	\item Do not wait to write
	\item Identify your key idea explicitly
	\item Tell a story paper structure
	\item Nail your contributions to the mast  introduction
	\item Related work
	\item Put your readers first the main body
	\item Listen to your readers
\end{enumerate}

\paragraph{Latex Writing Tips}

\begin{itemize}
	\item use \verb|\usepackage{microtype}| to improve the appearance of the text.
	\item use \verb|\tablename~\ref{tab}| to reference table.
	\item use \verb|\figurename~\ref{fig}| to reference figure.
	\item \verb|Eq.~\eqref{eq1}|
	\item comments from different people can be distinguished by different colors. \verb|\newcommand{\yl}[1]{{\color{blue}{[(YL): #1]}}}|
	\item use \verb|\newcommand{\method}{ABC\xspace}| to define model name or method
	\item mathematic notation \url{https://www.deeplearningbook.org/contents/notation.html}
\end{itemize}


\section{PhD Theses Tips}

\paragraph{The PhD Thesis Tips from \url{https://www.karlwhelan.com/Teaching/PhD/phd-writing-talk.pdf}}
What makes a good thesis? Forget about objectivity—beauty is in the eye of the beholder.
A good thesis is one that readers think is good.
So, you need to explain well what you are doing to somebody else.
Your ideas and results won’t sell themselves.
How you communicate your work is of crucial importance.
There is no point in having an interesting piece of research that nobody understands or sees the point of.
The “white coat” mentality—that we are mainly scientists who then do a write-up of our results—is misguided.
Writing is an essential part of the research process, not a last-minute thing to be rushed.
This is particularly true of PhD theses which are read very carefully by externs, who are hoping that you have explained what you have done in a clear fashion.

\paragraph{Writing Skills: More Important Than You Think}
What makes a good thesis? Forget about objectivity—beauty is in the eye of the beholder.
A good thesis is one that readers think is good.
So, you need to explain well what you are doing to somebody else Your ideas and results won’t sell themselves. How you communicate your work is of crucial importance.
There is no point in having an interesting piece of research that nobody understands or sees the point of.
The  ``white coat" mentality—that we are mainly scientists who then do a write-up of our results—is misguided.
Writing is an essential part of the research process, not a last-minute thing to be rushed.
This is particularly true of PhD theses which are read very carefully by externs, who are hoping that you have explained what you have done in a clear fashion.

\paragraph{Start by Avoiding Very Bad Writing}
People can be quite sensitive about their writing.
If you are a native English speaker and you have a degree, you probably think you know how to write and communicate well.
Chances are, you might be wrong.
Most MA graduates and even many professional PhD-qualified economists write very poorly.
Good writing is hard to define. Bad writing is easy to spot.
A badly-written thesis will have:
\begin{enumerate}
	\item Mis-spelled words.
	\item Missing words.
	\item Sentences that don’t make sense or aren’t proper sentences.
	\item Poor use of punctuation – full stops, commas etc.
\end{enumerate}
This annoys the reader because it makes things harder to read (and understand) but also because it’s so easy to prevent.
It suggests you didn’t take the time to be careful.

\paragraph{What To Do About It?}
Read, re-read, edit, and re-edit.
And do this as you go along.
Read and edit after you’ve written a page or so.
This can correct most of the common errors of style, grammar, and spelling that occur in the writing process.
Most of you actually do know what a sentence is.
In addition to catching typos, a quick re-read and edit allows you to check that what you’ve written gets across what you’ve been trying to say.
Sometimes you can think you’ve made a point clearly but then you read what you’ve written and it’s not so good.
Read your stuff aloud or slowly to yourself.
Does it sound right? Are you writing proper sentences? Are you over-using jargon or certain particular phrases?
Use spell checks but don’t rely on them.
Grammar. There are rules about how to use commas, colons, semi-colons, full stops, about what defines a sentence. Try to learn them.

\section{Abstract}



\section{Introduction}

\begin{enumerate}
	\item Introductions are crucial because they set up the reader to understand what your topic is and what you are going to do in your paper.
	\item Quickly explain two things:
	      \begin{enumerate}
		      \item Why is the topic of your paper interesting?
		      \item What did YOU do? What is YOUR contribution? A new question? An existing question but new methodology? Existing question, existing methodology, new data (e.g. no previous Irish application)?
	      \end{enumerate}

	\item Be willing to give an outline of what your results are but don’t get into too many details.
	\item Because of its importance, spend a lot of time on the introduction.
	\item But don’t make it too long. Three pages is a limit. Two is better.
	\item I often start writing the introduction as soon I have some results and then keep adjusting it as the paper evolves.
	\item Conclusions, on the other hand, should be kept short and to the point. Don’t repeat lots of stuff from the intro.
\end{enumerate}

\section{Literature Reviews}

\begin{enumerate}
	\item You need to explain your contribution.
	\item So it needs to be put in context.
	\item This will require discussion of previous studies in this area.
	\item Remember, though, the purpose is to set up your contribution, and distinguish it from previous work.
	\item Don’t simply provide a long list of separate descriptions of weakly related studies. (X (2007) did this. Y(2009) did that ...)
	\item Grouping studies together by type may be a better way of explaining it than listing lots of separate individual studies.
	\item A well-focused literature review is usually a useful component of a PhD thesis. It shows externs and your supervisor that you have read the literature.
	\item However, it is not essential that articles submitted for publication have a literature review. You may be able to summarise the existing literature in your introduction or work it into your opening section that sets up what you are doing.
\end{enumerate}

\section{Method}


\section{Result}



\section{Conclusion}
